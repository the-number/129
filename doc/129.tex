\documentclass[pdf]{beamer}
\mode<presentation>{
   \usetheme{Goettingen}
   \usecolortheme{seahorse}
%% \useinnertheme{inmargin}
%% \useoutertheme{infolines}
}
%% preamble
\title{$(10000001)_2$}
% \subtitle{with this byline}
\author{Gunter Liszewski}
\date{Belfast, August 2018}

\begin{document}

%% title frame
\begin{frame}
  \titlepage
\end{frame}

%% About this

\begin{frame}{About this}
  \begin{enumerate}[A]
    \pause
    \item\color<3->[rgb]{0.7,0.7,0.7} $(129)_{10}$, $(81)_{16}$, same thing, different looks
    \pause
    \item\color<3>[rgb]{0,0,0} What will be here?
    \pause
    \item\color<4>[rgb]{0,0,0} How?
    \pause
    \item\color<5>[rgb]{0.7,0.0,0.0} Thoughts!
%%  \end{itemize}
  \end{enumerate}
\end{frame}

%% one frame for one point
\begin{frame}{The point is this...}
  Because of this, there is that
$$\sum_{k=0}^{n}{k^2}={n(n+1)(2n+1)\over 6}$$
\end{frame}

%% example n=2

\begin{frame}{for example, $n=2$}
  Then $\sum_{0\le k\le2}k^2$ gives $0+1+4=5$, and on the other side $n=2$ and ${n(n+1)(2n+1)\over 6}$
sets as ${2(2+1)(2\times2+1)\over 6}$, or in concrete $2\times3\times5\over6$, or even just $5$.
\end{frame}

\begin{frame}{or on the other page}

  {\hfil\sl{}Fifteen men on the dead man's chest---}

             {\hfil\sl{}Yo-ho-ho, and a bottle of rum!}
\end{frame}

\end{document}
