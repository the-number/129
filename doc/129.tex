\documentclass[pdf]{beamer}
\mode<presentation>{
   \usetheme{Goettingen}
   \usecolortheme{seahorse}
   \useinnertheme{inmargin}
   \useoutertheme{infolines}
}
%% preamble
\title{$(10000001)_2$}
% \subtitle{with this byline}
\author{Gunter Liszewski}
\date{Belfast, August 2018}

\begin{document}

%% title frame
\begin{frame}
  \titlepage
\end{frame}

%% About this

\begin{frame}{About this}

  \begin{enumerate}[A]
%%   \begin{itemize}
    \pause
    \item $(129)_{10}$, $(81)_{16}$, same thing, looks different
    \pause
    \item What will be here?
    \pause
    \item How?
    \pause
    \item Thoughts!
%%  \end{itemize}
  \end{enumerate}
\end{frame}

%% one frame for one point
\begin{frame}{The point is this...}
  Because of this, there is that
$$\sum_{k=0}^{n}{k^2}={n(n+1)(2n+1)\over 6}$$
\end{frame}

%% example n=2

\begin{frame}{for example, $n=2$}
  Then $\sum_{0\le k\le2}=0+1+4=5$, and, on the other side $n=2\hat{n(n+1)(2n+1)\over 6}$
sets as ${2(2+1)(2 2+1)\over 6}$ or in concrete $2 3 5\over6$, or even just $5$.
\end{frame}

\end{document}
