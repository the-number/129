% Created 2019-05-21 Tue 13:42
\documentclass[11pt]{article}
\usepackage[utf8]{inputenc}
\usepackage[T1]{fontenc}
\usepackage{fixltx2e}
\usepackage{graphicx}
\usepackage{longtable}
\usepackage{float}
\usepackage{wrapfig}
\usepackage{soul}
\usepackage{textcomp}
%%% XXX: this package is not found by travis-ci: \usepackage{marvosym}
%%%  ditto: \usepackage{wasysym}
\usepackage{latexsym}
\usepackage{amssymb}
\usepackage{hyperref}
\tolerance=1000
\providecommand{\alert}[1]{\textbf{#1}}

\title{Testing emacs-lisp things}
\author{Gunter Liszewski}
\date{\today}
\hypersetup{
  pdfkeywords={},
  pdfsubject={},
  pdfcreator={Emacs Org-mode version 7.9.3f}}

\begin{document}

\maketitle


\href{file:///home/gunter/sbx/093-agile-org-mode/a/index.org}{\{Back to the index\}} \href{file:///home/gunter/sbx/093-agile-org-mode/b/test-this.org.html}{\{the source of this\}} \href{https://www2.bteco.ltd.uk/t/6/b/test-this.org}{\{download\}}

\section*{\textbf{TODO} (Backlog refinement) later, the team should review these methods and make choices}
\label{sec-1}
\subsection*{Learning and experiences}
\label{sec-1-1}

  The organisation should benefit from the experiences that are narrated here.
  However, the team should agree on a choice of tools/methods/approaches to be
  taken forward. The team should then focus only on the selected things. Other
  options should be kept options within reach, to facilitate review and reconsidering
  options.
\subsection*{This track, B has specific goals and the backlock refinement event should address them}
\label{sec-1-2}

  ATDD. Specify the products by giving concrete examples. Application tests should
  drive development forward. This track is about that. Is this agreed?
\subsection*{\textbf{TODO} The team should review and refine experiences gained earlier}
\label{sec-1-3}

  For example, a lates beamer thing.
  \href{https://the-number.github.io/129/doc/this-here.pdf}{\{this-here.pdf\}}
\section*{\textbf{TODO} The developer uses a template to formulate one specific test}
\label{sec-2}
\section*{\textbf{INPROGRESS} Find out about ERT}
\label{sec-3}

  How to run the ERT-TEST set.

\begin{verbatim}
   emacs -L some-where \
         -l ert \
         -L test \
         -l ert-tests.el \
         -f ert-summarize-tests-batch-and-exit output.log
\end{verbatim}

  Oh, how interesting. 24 and 26 \underline{are} different; so it seems.
  Does 26 drift towards cl? What about scope; dynamic, lexical\ldots{}
\begin{verbatim}
   emacs -L some-where \
    -l ert \
    -L test \
    -l ert-test.el \
    -f ert-run-tests-batch-and-exit |& tee _/$( date )-ert-tests
\end{verbatim}
\subsection*{\textbf{TODO} The stakeholders want to see the difference in lexical and dynamic, or else}
\label{sec-3-1}

   Is hygiene just about soap, the developers wonder.
   They need their curiosity entertained.
   Being free. Currying, or so. Again!

\begin{verbatim}
   run-ert-24
\end{verbatim}
\section*{\textbf{DONE} The developers should gain experience of using these tools}
\label{sec-4}


  Structure templates

  \verb.C-c C-,~ the sequence for ~(org-insert-structure-template).

\begin{verbatim}
(+ 1 101)
\end{verbatim}

  Macros


\begin{center}
\begin{tabular}{ll}
\hline
 Title:   &  Testing emacs-lisp things  \\
 Author:  &  Gunter Liszewski           \\
\hline
 File:    &  test-this.org              \\
 Time:    &  Tuesday, 21 May 2019       \\
\hline
\end{tabular}
\end{center}



  What effort this is


\begin{verbatim}
(org-set-effort &optional INCREMENT VALUE)
\end{verbatim}

And this is this! For now.
\section*{\textbf{DONE} To produce this, here}
\label{sec-5}
\subsection*{\textbf{DONE} The developers' and owner's machines may use directory locals to work this}
\label{sec-5-1}
\subsubsection*{\textbf{DONE} the developer should verify that these things work for \textbf{them}}
\label{sec-5-1-1}



\begin{verbatim}
((org-mode
  (eval progn (defun all-this ()
                                        ; (org-cycle '(64))
                (org-html-export-to-html)
                (with-current-buffer (htmlize-buffer)
                  (write-file (buffer-name (current-buffer)) nil)
                  (kill-buffer (current-buffer)))
                (defun push-to (a b c)
                  (shell-command
                   (format D
                           a b c)))
                (push-to (file-name-base) ".org.html" C)
                (push-to (file-name-base) ".html" C)
                (push-to (file-name-base) ".org" C)))
  (D . "scp -P 2222 %s%s www2.bteco.ltd.uk:/var/www/html/t/6/%s"))
 ("a/"
  (org-mode
   (C . "a/")))
 ("b/"
  (org-mode
   (C . "b/")))
 ("c/"
  (org-mode
   (C . "c/"))))
\end{verbatim}
\subsubsection*{\textbf{DONE} Now, on some other machine there may be different parameters.}
\label{sec-5-1-2}



\begin{verbatim}
((org-mode
  (eval progn (defun push-to (a b c)
     (shell-command
      (format D a b c))))
  (D . "cp -v %s%s /var/www/html/t/4/%s"))
 ("a/"
  (org-mode
   (C . "a/")))
 ("b/"
  (org-mode
   (C . "b/")))
 ("c/"
  (org-mode
   (C . "c/"))))
\end{verbatim}
\subsubsection*{\textbf{WAITING} The team should verify things for \textbf{their} workstation and \textbf{their} server.}
\label{sec-5-1-3}


 \verb,C-x ~C-e, may be hit at the end of each line.

\begin{verbatim}
 D
 (buffer-name (current-buffer))
 C
 (all-this)
\end{verbatim}
\subsection*{\textbf{DONE} On the way to get this done, this is a previous increment}
\label{sec-5-2}


  This is here for review. It is currently done via file local variables

  Within the source block hit \texttt{C-c C-c} to do all within.
  Or, to execute just a single SEXP, hit \texttt{C-x C-e} after it.

  Hit \texttt{C-c C-o} to open the following link that does \texttt{(all-this)}.
  \texttt{\{Do all of this\}} (after \texttt{all-this} if =defun=ed.)


\begin{verbatim}
(defun all-this ()
 "ALL-THIS, exported and htmlized and uploaded.

Note: this works on emacs 26.3. However, emacs 24.2 complains."
 (load-theme 'manoj-dark t nil)                    ;; C-x C-e
 (org-cycle '(64))                                 ;; showeverything
 ; (org-cycle '(4))                                ;; overview
 (org-html-export-to-html)                         ;; .html
 (with-current-buffer (htmlize-buffer) 
  (write-file (buffer-name (current-buffer)) nil)
  (kill-buffer (current-buffer)))                  ;; .org.html
q (defun push-to (a b c)
  "The org-mode basename, A, with extension, B, to its place withing, C"
  (shell-command
   (format
    "scp -P 2222 %s%s www2.bteco.ltd.uk:/var/www/html/t/4/%s"
    a b c)))                                      ;; abstract
 (push-to (file-name-base) ".org.html" "b/")
 (push-to (file-name-base) ".html" "b/")
 (push-to (file-name-base) ".org" "b/"))          ;; push
 (all-this)
\end{verbatim}

\begin{verbatim}
 (org-cycle '(4))                                ;; overview
 (org-cycle '(64))                               ;; showeverything
\end{verbatim}
\subsection*{\textbf{DONE} Another way To produce this, here: \verb~C-c C-o~ over \texttt{\{Do all of this\}}}
\label{sec-5-3}


  Within the source block hit \texttt{C-c C-c} to do all within.
  Or, to execute just a single SEXP, hit \texttt{C-x C-e} after it.

  Hit \texttt{C-c C-o} to open the following link that does \texttt{(all-this)}.
  \texttt{\{Do all of this\}} (after \texttt{all-this} if =defun=ed by the file local variables.)

\begin{verbatim}
   (org-cycle '(4))                                ;; overview
   (org-cycle '(64))                               ;; showeverything
   (a)
   (c)
\end{verbatim}

\begin{verbatim}
# Local Variables:
# mode: org
# eval: (progn (defun a ()
#         (format "%s" (buffer-file-name))))
# eval: (progn (defun c ()
#         "b/"))
# eval: (progn (defun all-this ()
#  (org-cycle '(64))
#  (org-html-export-to-html)
#  (with-current-buffer (htmlize-buffer) 
#   (write-file (buffer-name (current-buffer)) nil)
#   (kill-buffer (current-buffer)))
#  (defun push-to (a b c)
#   (shell-command
#    (format
#     "scp -P 2222 %s%s www2.bteco.ltd.uk:/var/www/html/t/4/%s"
#     a b c)))
#  (push-to (file-name-base) ".org.html" (c))
#  (push-to (file-name-base) ".html" (c))
#  (push-to (file-name-base) ".org" (c))))
# End:
\end{verbatim}
\section*{\textbf{DONE} This was here, this was the local thing}
\label{sec-6}

  and now, the last section in this file sets local variables. They facilitate
  the making of this.

\begin{verbatim}
(setq lexical-binding t)
;; 
lexical-binding
;; t
(add-file-local-variable 'mode 'org
)
;;
(add-file-local-variable 'mode 'fundamental)
;;
(add-file-
local-variable 'compile-command "ls")
;; 
(a)
;;
(all-this)
;;   
(desktop-save-in-desktop-dir)
;;
\end{verbatim}



\end{document}
